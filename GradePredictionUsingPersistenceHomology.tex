% Template for ISBI-2018 paper; to be used with:
%          spconf.sty  - ICASSP/ICIP LaTeX style file, and
%          IEEEbib.bst - IEEE bibliography style file.
% --------------------------------------------------------------------------
\documentclass{article}
\usepackage{spconf,amsmath,graphicx}

% Example definitions.
% --------------------
\def\x{{\mathbf x}}
\def\L{{\cal L}}

% Title.
% ------
\title{Application of Vectorized Persistence Homology Representations for 
Cancer Grading in Histopathology Images}
%
% Single address.
% ---------------
\name{Author(s) Name(s)\thanks{Thanks to XYZ agency for funding.}}
\address{Author Affiliation(s)}
%
% For example:
% ------------
%\address{School\\
%	Department\\
%	Address}
%
% Two addresses (uncomment and modify for two-address case).
% ----------------------------------------------------------
%\twoauthors
%  {A. Author-one, B. Author-two\sthanks{Thanks to XYZ agency for funding.}}
%	{School A-B\\
%	Department A-B\\
%	Address A-B}
%  {C. Author-three, D. Author-four\sthanks{The fourth author performed the work
%	while at ...}}
%	{School C-D\\
%	Department C-D\\
%	Address C-D}
%
% More than two addresses
% -----------------------
% \name{Author Name$^{\star \dagger}$ \qquad Author Name$^{\star}$ \qquad Author Name$^{\dagger}$}
%
% \address{$^{\star}$ Affiliation Number One \\
%     $^{\dagger}$}Affiliation Number Two
%
\begin{document}
%\ninept
%
\maketitle
%
\begin{abstract}
% 100 - 150 words

\end{abstract}
%
\begin{keywords}
Histopathology, Cancer Diagnosis, Cancer Grading, Persistence homology, Persistence diagrams, Persistence images, Persistence landscapes
\end{keywords}
%
\section{Introduction}
\label{sec:intro}
Cancer grading refers to the process of determining the degree of malignancy and is one of the primary criteria used in clinical practice to inform prognosis and plan the treatment of individual patients. However, achieving good reproducibility in grading most cancers remains one of the challenges in pathology practice.

\section{Background}
% introduce persistence homology, persistence diagrams
% talk about representational inconvenience of persistence diagrams for ML
% talk about persistence images
% talk about persistence landscapes
% describe how we apply them for cancer diagnosis and grading

\section{Method}


\section{Results}
We used the MICCAI 2015 Gland Segmentation Challenge Contest dataset~\cite{Sirinukunwattana2017GlandContest} to develop and validate the proposed method. This dataset contains a total of 165 images derived from 16 hematoxylin-eosin stained histological sections of stage T3 and T4 colorectal adenocarcinoma digitized using a Zeiss MIRAX MIDI SlideScanner with a pixel resolution of $0.620 \mu m$ equivalent to a 20x objective magnification. An expert pathologist delineated the boundary of all the glands in each image and graded the image as either $benign$ or $malignant$ based on the overall glandular architecture. Figure 1 shows representative example images of the two grades and Table 1 shows the distribution of grades in the training and testing sets.

% References
\bibliographystyle{IEEEbib}
\bibliography{mendeley}

\end{document}
