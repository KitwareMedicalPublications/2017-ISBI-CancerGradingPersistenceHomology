% Template for ISBI-2018 paper; to be used with:
%          spconf.sty  - ICASSP/ICIP LaTeX style file, and
%          IEEEbib.bst - IEEE bibliography style file.
% --------------------------------------------------------------------------
\documentclass{article}
\usepackage{spconf,amsmath,graphicx}

% Example definitions.
% --------------------
\def\x{{\mathbf x}}
\def\L{{\cal L}}

% Title.
% ------
\title{Application of Image-based Persistence Homology Descriptors for 
Cancer Diagnosis and Grading from Histopathology Images}
%
% Single address.
% ---------------
\name{Author(s) Name(s)\thanks{Thanks to XYZ agency for funding.}}
\address{Author Affiliation(s)}
%
% For example:
% ------------
%\address{School\\
%	Department\\
%	Address}
%
% Two addresses (uncomment and modify for two-address case).
% ----------------------------------------------------------
%\twoauthors
%  {A. Author-one, B. Author-two\sthanks{Thanks to XYZ agency for funding.}}
%	{School A-B\\
%	Department A-B\\
%	Address A-B}
%  {C. Author-three, D. Author-four\sthanks{The fourth author performed the work
%	while at ...}}
%	{School C-D\\
%	Department C-D\\
%	Address C-D}
%
% More than two addresses
% -----------------------
% \name{Author Name$^{\star \dagger}$ \qquad Author Name$^{\star}$ \qquad Author Name$^{\dagger}$}
%
% \address{$^{\star}$ Affiliation Number One \\
%     $^{\dagger}$}Affiliation Number Two
%
\begin{document}
%\ninept
%
\maketitle
%
\begin{abstract}
% 100 - 150 words

\end{abstract}
%
\begin{keywords}
Histopathology, Cancer, Diagnosis, Grading, Persistence homology, Persistence images, Persistence landscapes
\end{keywords}
%
\section{Introduction}
\label{sec:intro}
Cancer diagnosis and grading is the process of determining the presence and extent of malignancy and is one of the primary criteria used in clinical practice to inform prognosis and plan the treatment of individual patients. However, achieving good reproducibility in grading most cancers remains one of the challenges in pathology practice.
\section{Methods}

\section{Results}

\section{REFERENCES}
\label{sec:ref}

% References should be produced using the bibtex program from suitable
% BiBTeX files (here: strings, refs, manuals). The IEEEbib.bst bibliography
% style file from IEEE produces unsorted bibliography list.
% -------------------------------------------------------------------------
\bibliographystyle{IEEEbib}
\bibliography{refs}

\end{document}
